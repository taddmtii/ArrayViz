\documentclass{article}
\usepackage{graphicx} % Required for inserting images

\title{ArrayViz: An interactive array simulation for array traversal and manipulation of fundamental algorithms.}
\author{Tadd Trumbull II}
\date{September 2025}

\begin{document}

\maketitle

\begin{abstract}
Placeholder for my abstract..
\end{abstract}

\section{Introduction}
Array searching and sorting algorithms are foundational in computer science academia. Unfortunately, students often struggle to internalize on a deeper level these concepts when taught solely through pseudocode or static diagrams. ArrayViz allows for that gap to be diminished, providing an interactive, step by step simulation environment where students can manipulate array elements using custom parameters and index variables. Students will be able to observe the effect of comparisons and index swaps, and track the number of operations to intuitively grasp time complexity, a fundamental concept in understanding the time and space efficiency of an algorithm. By simulating algorithms at this low level, the project bridges the gap between theoretical abstraction and practical insight, making it easier for students to master fundamental algorithmic thinking. 
\section{Problem Statement}
Students often struggle to visualize these simple algorithms when the concept is first introduced, and traditional teaching methods typically present algorithms as abstract pseudocode, static diagrams or graphs. A prevalent challenge students face is a disconnect between abstract psuedocode and actual logic and execution. A student may know what swapping arr[i] with arr[j] means in psuedocode, but fail to picture the data shift itself programatically. In addition, instructors cannot show dynamic behavior through static lecture slides. These examples show the need to provide students with an interactive understanding of how these algorithms operate at a lower level through a visualization tool.
\section{Hypothesis}
ArrayViz aims to enhance students' fundamental understanding of array searching and sorting algorithms, and will result in a better understanding of the subject matter among students. If students use ArrayViz as a supplement to lectures and labs, their comprehension of algorithm steps, operational costs, trade-offs, etc... will improve compared to students who do not utilize the resource.
\section{Limitations}
Main limitations are a lack of reliable and consistent data and extensive prior research on how to accommodate different learning styles in a visual application.
\section{Delimitation's}
Delimitations include...

1. Focus on array based algorithms exclusively\\
2. Only core algorithms will be included such as...\\
- Linear Search\\
- Binary Search\\
- Bubble Sort\\
- Selection Sort\\
- Insertion Sort\\
- Merge sort\\
- Quicksort\\
-Does not teach syntax explicitly other than index variables\\
- English only\\
- Limited browser compatibility, with potential for explicit mobile support\\  
\section{Justification}
The need for an educational tool is clearly imperative, one that allows students to step through array algorithms in a guided manner that emphasizes the crucial role of index variables and array operations. ArrayViz aims to bridge the present gap between theory and implementation. Current tools like PythonTutor, which also visualize Python code, are broad in purpose and intent. ArrayViz serves a distinct purpose, and uses its own subset of Python for the applications specific use case. PythonTutor also does not have a gamification aspect, which incentivizes students. 
\section{Objectives}
- Supports custom index variables\\
- Step through algorithm process\\
- Comprehensive language documentation\\
- Database interaction to save students progress or visualizations \\
- Time complexity analysis (counting steps) \\
- Leaderboards \\
- Achievement system \\ 
- Gamification \\

\end{document}
